\documentclass[12pt, titlepage]{article}
\usepackage{graphicx}
\usepackage{amsmath}
\usepackage{tcolorbox}
\usepackage{parskip}
\usepackage[]{fancyhdr}
\usepackage{pgfplots}
\usepackage{mdframed}
\usepackage{multicol}
\pgfplotsset{compat=1.18}
\setlength{\headheight}{15pt}
\tcbuselibrary{breakable}

\title{Magnetic Fields and Electromagnetism}
\author{Matthew Pan}
\date{April 2025}

\newtcolorbox{Problem}{colback=blue!5!white,colframe=blue!50!black,title=Problem, breakable=true}

\begin{document}

\pagestyle{fancy}

\fancyhead{}
\fancyhead[L]{Pan}
\fancyhead[R]{\thepage}

\maketitle

\section*{12.1 Magentic Fields}

A magnetic field is a vector field that describes the magnetic force exerted on moving charges, electric currents, or magnetic materials. Since magnetic fields are created by magnetic dipoles, which always have a north and a south pole, magnetic field lines always form a closed loop: north to south outside a magnet and south to north inside a magnet.

Because magnetic poles always occur as dipoles, the net magnetic field through any closed surface will always be zero:
\begin{equation*}
    \oint B \cdot dA = 0
\end{equation*}

\end{document}