\documentclass[12pt, titlepage]{article}
\usepackage{graphicx}
\usepackage{amsmath}
\usepackage{tcolorbox}
\usepackage{parskip}
\usepackage[]{fancyhdr}
\usepackage{pgfplots}
\usepackage{mdframed}
\pgfplotsset{compat=1.18}
\setlength{\headheight}{15pt}
\tcbuselibrary{breakable}

\title{Electric Circuits}
\author{Matthew Pan}
\date{March 2025}

\newtcolorbox{Problem}{colback=blue!5!white,colframe=blue!50!black,title=Problem, breakable=true}

\begin{document}

\pagestyle{fancy}

\fancyhead{}
\fancyhead[L]{Pan}
\fancyhead[R]{\thepage}

\maketitle

\section*{11.1 Electric Current}

Electric current measures the number of coulombs of charge that move through an area in a given time. The unit of electric current is the ampere, or coulombs per second.
\begin{equation*}
    I = \frac{Q}{\Delta t} = \frac{dq}{dt}
\end{equation*}

We can use the hydraulic analogy to make understanding electric currents a little easier. 

\end{document}