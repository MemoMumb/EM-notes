\documentclass[12pt, titlepage]{article}
\usepackage{graphicx}
\usepackage{amsmath}
\usepackage{tcolorbox}
\usepackage{parskip}
\usepackage[]{fancyhdr}
\usepackage{pgfplots}
\usepackage{mdframed}
\pgfplotsset{compat=1.18}
\setlength{\headheight}{15pt}
\tcbuselibrary{breakable}

\title{Conductors and Capacitors}
\author{Matthew Pan}
\date{March 2025}

\newtcolorbox{Problem}{colback=blue!5!white,colframe=blue!50!black,title=Problem, breakable=true}

\begin{document}

\pagestyle{fancy}

\fancyhead{}
\fancyhead[L]{Pan}
\fancyhead[R]{\thepage}

\maketitle

\section*{10.1 - Electrostatics with Conductors}

\begin{itemize}
    \item Insulators - materials in which electrons are not able to move freely 
    \item Conductors - materials in which electrons are able to move freely
    \begin{itemize}
        \item When charge is added to a conductor, the electrons will spread out as much as possible over the surface of the object, which occurs in a negligible amount of time. The charge inside the object will zero.
        \item In a solid conductor with a uniform surface (a surface with a constant curvature such as a sphere or a plane), charge will be uniformly distributed over the surface.
        \item In a solid conductor with a nonuniform surface, the charge density will increase as the surface becomes more curved.
    \end{itemize}
\end{itemize}

In the real world, this property can be used to gradually discharge objects, preventing charge from building up. 

\section*{10.2 - Redistribution of Charge Between Conductors}

In conductors, charge may be transferred through conduction, when two objects are touching, or induction, when two objects are near each other but not touching. 

In E\&M, "ground" is a reference point where electric potential is zero. It can also be an infinite source or receptacle for electrons. For example, if a positively charged object were said to be grounded, electrons would flow from ground to the object, making it neutral. The opposite would happen if the object were negative: electrons would flow from the object to ground, resulting in the object still being neutral.  

\begin{Problem}
    An insulating sphere with radius $a$ has a uniform charge density with a total charge of $+5q$. The sphere is surrounded by a conducting shell with inner radius $b$ and outer radius $c$. The shell carries an overall charge $-2q$.

    Determine the charge:
    \begin{itemize}
        \item[] a) Inside the conductor, between radii $b$ and $c$
        \item[] b) On the inner surface of the conducting shell
        \item[] c) On the outer surface of the conducting shell
    \end{itemize}

    \tcblower

    \textit{Solution. }
    \begin{itemize}
        \item[] a) The charge inside a conductor is always 0.
        \item[] b) The conducting shell will become polarized, drawing electrons to the inner surface to counter the positive charge from the sphere. The answer is $-5q$.
        \item[] c) Since the shell has a $-2q$ net charge and the charge inside a conductor is 0, the charge on the outer surface must be $+3q$.
    \end{itemize}
\end{Problem}

\section*{10.3 - Capacitors}

Capacitors are used to store energy by moving charges to two plates with an applied potential difference. When a capacitor is connected to a battery, electrons on the plate connected to the positive terminal are pulled away, leaving a positive charge, while the negative terminal pushes electrons towards the other plate, giving it a negative charge. When the battery is removed, the capacitor remains in its charged state until a load is connected (for example, an LED lightbulb), following which the electrons flow from the negatively charged plate to the positively charged plate, illuminating the bulb.

The capacitance of a capacitor is defined as 
\begin{equation*}
    C= \frac{Q}{\Delta V}
\end{equation*}

measured in farads (F), or coulombs per volt.

The magnitude of electric field of a parallel plate capacitor can be determined using Gauss's Law, where A is the area of a plate in the capacitor. 
\begin{align*}
    \oint E \cdot dA &= \frac{q_{enc}}{\epsilon_0} \\
    E \cdot A &= \frac{Q}{\epsilon_0} \\
    E &= \frac{\sigma}{\epsilon_0}
\end{align*}

\subsection*{Capacitance of Various Capacitor Geometries}

Solving for the capacitance of a capacitor involves three steps:

\begin{enumerate}
    \item Find the electric field between the two surfaces with Gauss's Law
    \item Find $|\Delta V|$ by integrating the electric field over the distance between the two surfaces.
    \item Substitute the derived $|\Delta V|$ into the equation for capacitance: $C = \frac{Q}{\Delta V}$
\end{enumerate}

\subsubsection*{Parallel Plate Capacitor}

\textbf{Step 1: Electric Field}

We retain the previously derived equation for electric field: $E = \frac{\sigma}{\epsilon_0}$

\textbf{Step 2: Voltage}

Integrating E:
\begin{align*}
    \Delta V & = - \int_{a}^{b}E \, dx \\
    &= - \int_{a}^{b} \frac{\sigma}{\epsilon_0} \, dx\\
    &= -\frac{\sigma}{\epsilon_0} \int_{a}^{b} \, dx\\
    &= -\frac{\sigma}{\epsilon_0} (b-a)
    = -\frac{\sigma \, d}{\epsilon_0}
    = -\frac{Q \, d}{A \, \epsilon_0}
\end{align*}

\textbf{Step 3: Substitution}
\begin{equation*}
    C = \frac{\epsilon_0 \, A}{d}
\end{equation*}

\subsubsection*{Coaxial Cable Capacitor}

Where $r_a$ is the radius of the inner cylinder, $r_b$ is the radius of the outer shell, and $L$ is the length of the capacitor.

\textbf{Step 1: Electric Field}
\begin{align*}
    &E \, (2\pi r L) = \frac{Q}{\epsilon_0} \\
    &E = \frac{Q}{2 \pi \epsilon_0 r L}
\end{align*}

\textbf{Step 2: Voltage}

Integrating E:
\begin{align*}
    \Delta V & = - \int_{a}^{b}E \, dx \\
    &= - \int_{r_a}^{r_b} \frac{Q}{2 \pi \epsilon_0 r L} \, dr\\
    &= - \frac{Q}{2 \pi \epsilon_0 L} \int_{r_a}^{r_b} \frac{1}{r} \, dr\\
    & = - \frac{Q (\ln{r_b}-\ln{r_a})}{2 \pi \epsilon_0 L}
    = - \frac{Q}{2 \pi \epsilon_0 L}\ln{\frac{r_b}{r_a}}
\end{align*}

\textbf{Step 3: Substitution}
\begin{equation*}
    C = \frac{2 \pi \epsilon_0 L}{\ln{\frac{r_b}{r_a}}}
\end{equation*}

\subsubsection*{Concentric Sherpicial Capacitor}

Where $r_a$ is the radius of the inner sphere and $r_b$ is the radius of the outer sphere.

\textbf{Step 1: Electric Field}
\begin{align*}
    &E \, (4\pi r^2) = \frac{Q}{\epsilon_0} \\
    &E = \frac{Q}{4 \pi \epsilon_0 r^2}
\end{align*}

\textbf{Step 2: Voltage}

Integrating E:
\begin{align*}
    \Delta V & = - \int_{a}^{b}E \, dx \\
    &= - \int_{r_a}^{r_b} \frac{Q}{ 4\pi \epsilon_0 r^2} \, dr\\
    &= - \frac{Q}{4 \pi \epsilon_0} \int_{r_a}^{r_b} \frac{1}{r^2} \, dr\\
    & = - \frac{Q}{4 \pi \epsilon_0}(\frac{1}{r_b}-\frac{1}{r_a})
\end{align*}

\textbf{Step 3: Substitution}
\begin{align*}
    C &= \frac{4 \pi \epsilon_0}{\frac{1}{r_b}-\frac{1}{r_a}}\\
    &= 4 \pi \epsilon_0(\frac{r_{a}r_{b}}{r_{b}-r_a})
\end{align*}

\begin{Problem}
    Two plates of a fully charged parallel plate capacitor are moved closer together while a 9V battery is connected. Does the magnitude of the electric field increase, decrease, or remain constant? 
    \tcblower
    \textit{Solution. } Increases. Since the voltage remains constant and distance decreases, according to the equation $E=\frac{\Delta V}{d}$, $E$ increases.
\end{Problem}

\subsection*{Potential Energy of Capacitors}

The potential energy stored by a capacitor can be represented three ways:
\begin{equation*}
    U_C=\frac{Q^2}{2C}=\frac{1}{2}C(\Delta V)^2=\frac{1}{2}Q\Delta V
\end{equation*}

\newpage

\section*{10.4 - Dielectrics}

All dielectrics are instulators, but not all insulators are good dielectrics. A dielectric is a material that cannot conduct electricity but can be polarized when placed in an electric field. When a dielectric is placed in an electric field, the atoms in the dielectric become polarized, creating their own weaker electric field opposite the capacitor that reduces the overall electric field. Since electric field and voltage are related $E = Vd$, a decrease in $E$ results in a decrease in $V$. By the formula $C=\frac{Q}{\Delta V}$, a decrease in $\Delta V$ results in an increase in $C$.

Filling a capacitor with a dielectric material with dielectric constant $\kappa$ increases its capacitance by a factor of $\kappa$, that is 
\begin{equation*}
    C = \kappa \epsilon_0 \frac{A}{d} 
\end{equation*}
for a parallel plate capacitor.

\end{document}