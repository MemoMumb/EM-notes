\documentclass[12pt, titlepage]{article}
\usepackage{graphicx}
\usepackage{amsmath}
\usepackage{tcolorbox}
\usepackage{parskip}
\usepackage[]{fancyhdr}
\usepackage{pgfplots}
\usepackage{mdframed}
\pgfplotsset{compat=1.18}
\setlength{\headheight}{15pt}
\tcbuselibrary{breakable}

\title{Conductors and Capacitors}
\author{Matthew Pan}
\date{March 2025}

\newtcolorbox{Problem}{colback=blue!5!white,colframe=blue!50!black,title=Problem, breakable=true}

\begin{document}

\pagestyle{fancy}

\fancyhead{}
\fancyhead[L]{Pan}
\fancyhead[R]{\thepage}

\maketitle

\section*{10.1 - Electrostatics with Conductors}

\begin{itemize}
    \item Insulators - materials in which electrons are not able to move freely 
    \item Conductors - materials in which electrons are able to move freely
    \begin{itemize}
        \item When charge is added to a conductor, the electrons will spread out as much as possible over the surface of the object, which occurs in a negligible amount of time. The charge inside the object will zero.
        \item In a solid conductor with a uniform surface (a surface with a constant curvature such as a sphere or a plane), charge will be uniformly distributed over the surface.
        \item In a solid conductor with a nonuniform surface, the charge density will increase as the surface becomes more curved.
    \end{itemize}
\end{itemize}

In the real world, this property can be used to gradually discharge objects, preventing charge from building up. 

\section*{10.2 - Redistribution of Charge Between Conductors}

In conductors, charge may be transferred through conduction, when two objects are touching, or induction, when two objects are near each other but not touching. 

In E\&M, "ground" is a reference point where electric potential is zero. It can also be an infinite source or receptacle for electrons. For example, if a positively charged object were said to be grounded, electrons would flow from ground to the object, making it neutral. The opposite would happen if the object were negative: electrons would flow from the object to ground, resulting in the object still being neutral.  

\begin{Problem}
    An insulating sphere with radius $a$ has a uniform charge density with a total charge of $+5q$. The sphere is surrounded by a conducting shell with inner radius $b$ and outer radius $c$. The shell carries an overall charge $-2q$.

    Determine the charge:
    \begin{itemize}
        \item[] a) Inside the conductor, between radii $b$ and $c$
        \item[] b) On the inner surface of the conducting shell
        \item[] c) On the outer surface of the conducting shell
    \end{itemize}

    \tcblower

    \textit{Solution. }
    \begin{itemize}
        \item[] a) The charge inside a conductor is always 0.
        \item[] b) The conducting shell will become polarized, drawing electrons to the inner surface to counter the positive charge from the sphere. The answer is $-5q$.
        \item[] c) Since the shell has a $-2q$ net charge and the charge inside a conductor is 0, the charge on the outer surface must be $+3q$.
    \end{itemize}
\end{Problem}

\end{document}